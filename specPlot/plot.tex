% Plot quick method using pgfplots to plot spect
%
% Copyright (C) 2014 Alex J. Grede
% GPL v3, See LICENSE.txt for details
% This function is part of specPlot

\documentclass[tikz]{standalone}
\usepackage{stix}
\usepackage{pgfplots}
\usepackage{amsmath}
\usepackage{xfrac}
\usepackage{bm}
\usepackage[use-xspace=true,
redefine-symbols=false,
range-units=single,
inter-unit-product=\ensuremath{{}\cdot{}}]{siunitx}
\sisetup{detect-all}
\edef\dtapth{\jobname.csv}
\DeclareSIUnit\decade{decade}
\DeclareSIUnit\arb{a.u.}

\pgfplotsset{
  compat=1.11,
  width=8.8cm,
  label style={font=\large},
  tick label style={font=\normalsize},
  every axis/.append style={
    very thick,
    line cap=rect,
    tick style={very thick,line cap=rect,black}
  },
  every axis plot/.append style={thick}
}

\definecolor{color0}{HTML}{5C8540}
\definecolor{color1}{HTML}{7D6BCF}
\definecolor{color2}{HTML}{BE5F2F}
\definecolor{color3}{HTML}{66BCBA}
\definecolor{color4}{HTML}{8AD643}
\definecolor{color5}{HTML}{44496F}
\definecolor{color6}{HTML}{D0B648}
\definecolor{color7}{HTML}{97A1D0}
\definecolor{color8}{HTML}{454324}
\definecolor{color9}{HTML}{7ED98F}

\pgfplotscreateplotcyclelist{mycolor}{%
  color0,color1,color2,color3,color4,color5,color6,color7,color8,color9}

\begin{document}
\begin{tikzpicture}
  \begin{axis}[
    scale only axis,
    axis x line*=bottom,
    cycle list name = mycolor,
    xmin=(( limits['xmin'] )),xmax=(( limits['xmax'] )),
    xtick = {(( ticks['xmajor'] ))},
    xticklabels = {(( ticks['xlables']))},
    minor xtick={(( ticks['xminor'] ))},
    xlabel={\textbf{%
        (( labels['xlabel']['text'] ))%
        (< if labels['xlabel']['symbol'] is defined >)
        \(\bm{\left( (( labels['xlabel']['symbol'] )) \right)}\)
        (< endif >)
        (< if labels['xlabel']['units'] is defined >)
        \(\bm{\left[
            (< if labels['xlabel']['mult'] is defined >)
            \times
            \protect{\SI{(( labels['xlabel']['mult'] ))}{(( labels['xlabel']['units'] ))}}
            (< else >)
            \protect{\si{(( labels['xlabel']['units'] ))}}
            (< endif >)
          \right]}\)
      }},
    ymin=(( limits['ymin'] )),ymax=(( limits['ymax'] )),
    ytick = {(( ticks['ymajor'] ))},
    yticklabels = {(( ticks['ylables']))},
    minor ytick = {(( ticks['yminor'] ))},
    ylabel={\textbf{%
        (( labels['ylabel']['text'] ))%
        (< if labels['ylabel']['symbol'] is defined >)
        \(\bm{\left( (( labels['ylabel']['symbol'] )) \right)}\)
        (< endif >)
        (< if labels['ylabel']['units'] is defined >)
        \(\bm{\left[
            (< if labels['ylabel']['mult'] is defined >)
            \times
            \protect{\SI{(( labels['ylabel']['mult'] ))}{(( labels['ylabel']['units'] ))}}
            (< else >)
            \protect{\si{(( labels['ylabel']['units'] ))}}
            (< endif >)
          \right]}\)
      }},
    legend style = {
      at={(axis cs:(( legend['position'] )))},
      anchor=(( legend['anchor'] )),
      font={\footnotesize},
      cells={anchor=west},
      draw=none}
    ]

    (< for idx,leg in enumerate(legend)  >)

    \addplot+[no markers, solid, smooth]
    table [x index=0,y index=(( idx+1 )),col sep=comma]{\dtapth};

    \addlegendentry{(( leg ))}

    (< endfor >)
  \end{axis}
  \begin{axis}[
    scale only axis,
    axis x line*=top,
    xmin=(( limits['xmin'] )),xmax=(( limits['xmax'] )),
    xtick = {(( ticks['x2major'] ))},
    minor xtick = {(( ticks['x2minor'] ))},
    xticklabels = {(( ticks['x2lables']))},
    xlabel={\textbf{%
        (( labels['x2label']['text'] ))%
        (< if labels['x2label']['symbol'] is defined >)
        \(\bm{\left( (( labels['x2label']['symbol'] )) \right)}\)
        (< endif >)
        (< if labels['x2label']['units'] is defined >)
        \(\bm{\left[
            (< if labels['x2label']['mult'] is defined >)
            \times
            \protect{\SI{(( labels['x2label']['mult'] ))}{(( labels['x2label']['units'] ))}}
            (< else >)
            \protect{\si{(( labels['x2label']['units'] ))}}
            (< endif >)
          \right]}\)
      }},
    ymin=(( limits['ymin'] )),ymax=(( limits['ymax'] )),
    hide y axis]
  \end{axis}
\end{tikzpicture}
\end{document}

%%% Local Variables:
%%% mode: latex
%%% TeX-master: t
%%% End:
